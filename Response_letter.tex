\documentclass{letter}
\usepackage{graphicx} % Required for inserting images
\usepackage[dvipsnames]{xcolor}

\newcommand{\auth}[1]{\textcolor{RoyalBlue}{#1}}
\newcommand{\MF}[1]{\textcolor{red}{#1}}

\begin{document}

Dear editors of the Journal of School Psychology,

We hereby submit our revision of manuscript XXX.

[TODO: Thank editors and reviewers]

Below, you will find our point-by-point response to the reviewer's comments, including changes made to the manuscript. Reviewer's comments are printed in black, our response printed in blue, verbatim parts from the manuscript are italicized. 

Sincerely,

Dr. Marjolein Fokkema, also on behalf of Dr. Mirka Henninger & Prof. Carolin Strobl


\textbf{Reviewer 1}

This paper is a hands-on article providing two examples on how to use model-based recursive partitioning (MOB) in educational research. Two MOB methods were presented: (1) MOB for mixed-effects models and (2) MOB for Rasch measurement and Item Response Theory models. The provided examples are well presented; R-scripts and data are explained well and provided through a github link.

\auth{We thank the reviewer for evaluating our manuscript, the positive evaluation and the helpful suggestions.}

The authors should provided an additional references for the interested reader in the journal called Psychological Test and Assessment Modeling, volume 61, issue 4, 2019; special issue on "Tree-based methods for regression and classification - statistical methods at the interface of graphics and statistics."

\auth{We thank the reviewer for pointing out this special issue, which is indeed very relevant for the current paper. In the first paragraph of the Introduction, we have included three references to papers from the special issue:} 

\auth{\textit{For example, recursive partitioning methods have been used to risk factors for adolescent crime (Fritsch, Haupt, Lösel \& Stemmler, 2019) and corporal punishment (Stemmier, Heine \& Wallner, 2019), and to identify predictors of math ability (Ding \& Zhao, 2019).}}


Minor issues:

- Please spell out acronyms like GLMM, DIF and GLM when being introduced for the first time.

\auth{The manuscript now explains the acronym GLMM (in Abstract and Introduction). We have replaced the acronym GLM with "generalized linear model" in the Introduction, because it was only used once. We double checked for the other acronyms (IRT, DIF) whether they were explained upon first use.}

- On page 9, lower paragraph, there are three hints, that there will be an 'Application Example 2'

\auth{Indeed, the many pointers made the text on page 9 somewhat repetitive. We have eliminated some pointers, leaving only two, of which one is in round brackets.}

- p. 11 lower paragraph, there is a period missing between 'height' and 'Note'.

\auth{Thanks for pointing this out, this has now been fixed.}

- p. 18, second paragraph; see the word 'notcontain' needs to be separated.

\auth{Thanks for pointing this out, this has now been fixed.}

- in Figure 6 and 7; there is a node 4 but no node 3!?

\auth{There is a node 4 in these Figures. Nodes 1, 2 and 4 are inner nodes, their numbers are indicated in a square box at the top of the node. Nodes 3, 5 and 6 are terminal nodes, their numbers are written above the node, without a box around it.}

\MF{I can understand where the confusion comes from, but assume that readers will be able to spot node 4, even after some initial confusion. Or should we add a note to the Figures?}


\textbf{Reviewer 2}

In this article, the authors present a tutorial-style introduction of model-based partitioning (MOB). From a person-oriented perspective, this method is of great importance. This perspective proceeds from the assumption that parameters that are estimated for large entities of individuals (populations) can rarely, if ever, be generalized to the individual. Therefore, this article is most welcome, as it introduces readers to a very useful methodology. Both, the introductory section and the application section, are correct, easy to read, and fulfill, thus, the expectations readers may have when processing a tutorial. Unfortunately, however, there are a few issues that need to be addressed before this article can be published.

\auth{We thank the reviewer for evaluating our manuscript, the positive evaluation and the helpful suggestions.}

1. The first of these issues concerns the introduction into MOB. When writing a tutorial, authors need to specify a target audience. In this article, this audience is not easily identified. On the one hand, the text of this article is completely non-technical, in the sense that it contains no equations or formal model specifications. On the other hand, however, this text contains a large number of technical terms that readers in need of a tutorial may not be familiar with. Therefore, I recommend considering to also include a formal presentation of the models that are discussed in the intro and the examples. Those readers who are familiar with the technical terms may appreciate a formal presentation of these models.

\auth{TODO}

\MF{Not completely sure how to deal with this. Adding ALL model representations (parameter stability tests, GLMM, Rasch models) to the paper makes it unsuitable for the readership, I think. We could add it as an Appendix, of course. But I think will not be very relevant for the readership.}

2. The second of these issues concerns the results of the two application examples. In neither example, the resulting subgroups are sufficiently described or explained. This is an omission that lowers the appeal of the tutorial. Please complete the description of the results.

\auth{We have added a heading \textit{Interpretation of a GLMM tree} to the first application example, and a heading \textit{Interpretation of a Rasch tree} to the second application example, to more clearly indicate where we show how to interpret the splits and subgroups in the tree. }

\MF{I have difficulty seeing where this remark cames from. A whole page was dedicated to explaining the tree in Figure 3. Around Figures 5\& 6, there is also an explanation of the subgroups and corresponding difficulty levels for the items. I added some headings to both application examples, so they have similar subsections: Dataset; Fitting a GLMM/Rasch tree; Pruning a Rasch tree; Interpreting a GLMM/Rasch tree. Feel free to correct / undo. And I added a few minor clarifications to the "Interpreting a Rasch tree section" for Figures 6 and 7. Please double check my addition to the explanation of Figure 7, as I added Age being $>24$ in the description of node 5. Should we simply point the reviewer to page 12-13 for interpretation of the GLMM tree, and to page 20-24 for interpretation of Rasch trees? OR copy it here? We coul do the latter, and simply state that we described the subgroups in 'more detail' in the current version of the manuscript.}


3. Finally, before resubmitting, please do some language editing. There are multiple grammatical errors and incomplete sentences in the text. Please correct.

\auth{We apologize for the mistakes, language errors can make reviewing a more tedious job. We have carefully corrected errors throughout the manuscript.}

\MF{I fixed some issues, but still have to do a final round of language checks.}


\textbf{Reviewer 3}

Thank you for the opportunity to review this team's interesting methodological work. The authors present a didactic paper focused on describing the practical features of the semi-parametric model based recursive partitioning method and two ways to apply it in exploratory school psychology-relevant research. I commend the authors on writing so clearly about such a complicated topic enhancing its value for applied researchers without a rigorous background in quantitative methodology. I am also grateful that accompanying code and data were provided. I spent some time replicating the analyses and trying it out on some of my own data sets. I believe the paper has promise as a useful tutorial for this audience but ask that the authors consider some areas of improvement I noted while reading.

\auth{We thank the reviewer for evaluating our manuscript, the positive evaluation and the helpful suggestions.}

1.      In the MOB Algorithm section, I think it would be clearer for readers less familiar with CART, MOB, and related methods if actual example variable names accompanied the abstract y, x1, etc. labels since this is the first exposure. For instance, "...A continuous response variable y (e.g., total difficulties score from the Strengths and Difficulties Questionnaire), two continuous covariates x1 (e.g., Age) and x3 (e.g., Reading comprehension scores), and one categorical covariate x2 (e.g., binary gender)..." You obviously go on to give good concrete examples later, but I still think something like this would help the concepts "click" faster earlier on.

\auth{Thank you for this recommendation, which we have adopted on page 3-4:}

\auth{\textit{We illustrate the idea using a simple, simulated toy dataset, comprising 250 observations and four variables: A continuous response variable $y$ (e.g., a total score for behavioral difficulties), and three covariates, $x_1$ (e.g., age), $x_2$ (e.g., reading comprehension score) and $x_3$ (e.g., gender, with levels male, female and non-binary), as possible partitioning variables.}}

\auth{We again use it on page 5-6:}

\auth{\textit{The means (intercepts) in the terminal nodes are 4.42, 4.78 and 9.17, respectively. While the means and distributions in nodes 2 and 4 appear very similar, the mean in node 5 is obviously higher. Substantively, the tree suggests and interaction effect: The effect of $x_2$ depends on the level of $x_1$ (vice versa). If $x_1$ were age and $x_2$ gender, this would indicate that gender differences only start to occur at somewhat later age.}}


2.      I know it is well beyond the scope of the paper to present all the technical detail of how the math works behind the scenes yet I think adding a little more about how "instabilities" are quantified broadly would be instructive (like how residual sums of squares are compared in CART).

\auth{TODO, and connect to the first comment of reviewer 2.}

\MF{I'd like your input on how to best deal with this remark. It would be easy to provide some formula's for EFPs, but I also think this will generally not be helpful for the readership.}

3.      One of the stopping criteria was too few observations within a node (i.e., < 250). But it was not clear how 250 was determined. How small is too small? Please add more comment on considerations for this.

\auth{We have now explained this in more detail on page 11 of the manuscript:}

\auth{\textit{Finally, to aid interpretability we want to retain large enough subgroups, we specified the minsize argument. Splits will only be implemented if the resulting nodes contain at least 250 observations. With a total sample size of 1433, even small differences in the effects of age and Head Start participation between subgroups may become significant. The average number of measurements per family was 5.6.
A minimum node size of 250 thus ensured that the estimated effects of age and Head Start
participation in the terminal nodes would at least be based on data from about 50 families.}.}

4.      Non-significant p values were another stopping criterion. Are the p values adjusted for multiple exploratory comparisons so as not to overfit?

\auth{Yes, a Bonferroni correction is applied to the $p$ values, to account for testing instability with respect to all of the potential partitioning variables. We have now made this explicit on page 5 of the manuscript:}

\auth{\textit{By default, a Bonferroni correction is applied to the $p$ values, to account for the fact that the tests are performed for all of the potential partitioning variables.}}

5.      Given that the point of the method is to explore data and generate hypotheses that could be tested with future samples, I think the key thing missing from the paper is some further demonstration of what follow up might actually look like based on these results with real, simulated, or even completely hypothetical data. While there are references at the end pointing to other work on enhancing the stability of results, cross-validation techniques, etc., these papers are more technical in nature and do not serve the same didactic function as this paper. Without going into great technical detail, showing the larger research context around how results from the MOB method could fruitfully inform future studies would add tremendous value for the (mostly) applied researchers who read the journal. Although certainly more work on your part, I truly think adding this additional element is entirely reasonable for a tutorial paper and more in line with the goals of the journal.

\auth{Thank you for pointing this out, we agree with the reviewer's comment, and have added the following suggestion }

\auth{\textit{It should be noted that MOB and GLMM trees are exploratory techniques: The subgroup structures are detected from the data. Such data-driven procedures cannot provide valid standard errors and significance tests to evaluate significance of the observed subgroup differences. To that end, the subgroup structure that was found should be used in a confirmatory analysis on a new sample, that was not used to fit the tree. This would allow to test the hypothesis, either using frequentist or Bayesian approaches, of whether the parameters of interest really differ between subgroups. Researchers who want to both detect subgroups and test hypotheses on the parameter differences on one dataset, are advised to split their dataset into two parts before the analyses. Then one part of the dataset can be used for exploration, and the other part can be used to test the parameter differences between subgroups.}}

\MF{This is just a starting point. I think I need to make this more concrete. Suggestions are welcome.}

6.      I think the paper would flow better if the general description sections for the two applications (e.g., Using MOB for subgroup detection in mixed-effects models) would come right before their respective example walkthrough rather than split up. As it stands now the two general descriptions are grouped together followed by the two examples which required some backtracking on my part to recall the important parts of each application.

\auth{Thank you for this recommendation. We agree with the reviewers' comment and have restructured the manuscript accordingly. \textit{Application Example 1} is now included as a subsection of the section \textit{MOB for Subgroup Detection in Mixed-Effects Models}. Similarly, \textit{Application Example 2} is now included as a subsection of the section \textit{MOB for Detecting DIF in Measurement Models}.}



\end{document}
